\documentclass[titlepage]{article}
\usepackage{geometry}
\geometry{a4paper, total={170mm,257mm},left=20mm, top=10mm,}
\usepackage[colorlinks=true,linkcolor=blue,urlcolor=black]{hyperref}
\usepackage{bookmark}
\usepackage{graphicx}
\usepackage{titling}
\graphicspath{ {images/} }

\pretitle{%
  \begin{center}
  \LARGE
    \includegraphics[width=6cm]{resim.eps}\\[\smallskipamount]
}
\posttitle{\end{center}}
\begin{document}
\title {%
  RESim Remote Access Guide \\
  \large Configuring RESim servers and clients}
\maketitle
\section{Introduction}

This guide provides instructions for accessing RESim remotely on a RESim server

These instructions assume your remote RESim server is bladet9.  Adjust as needed based on the
server assigned to you.

\section{Client setup}
Perform these steps on the computer that you will use to access RESim.  These instructions
assume a Linux host upon which IDA Pro is installed, e.g., a VM.

Use {\tt ssh-keygen} to create an ssh key pair and send the public key to mfthomps@nps.edu
along with a requested user ID.  You will then receive an assigned server name and IP address.
After receiving the reply, do the following:

\begin{itemize}

\item Add the assigned blade and IP address to your /etc/hosts:
\begin{verbatim}
    10.20.200.159 bladet9
\end{verbatim}

\item Set up your ~/.ssh/config file to identify the gateway and RESim server.  An example
assuming your assigned RESim server is bladet9.  Replace that, and the ProxyCommand IP with 
those assigned to you.  Also replace the IdentifyFile with your own,
and replace both User fields with your assigned login ID.:

\begin{verbatim}
Host cgc-gw
   IdentityFile        ~/.ssh/id_mfthomps_gfe
   User                mfthomps
   ServerAliveInterval 1
   ServerAliveCountMax 60
   TCPKeepAlive        yes
   HostName            205.155.65.172

Host bladet9
  HostName             bladet9
  User                 mike
  IdentityFile         ~/.ssh/id_mfthomps_gfe
  ProxyCommand         ssh -q cgc-gw nc 10.20.200.159 22
\end{verbatim}

\item Create an ssh agent on your local Linux.  E.g., source this script (using
your ssh id file):

\begin{verbatim}
   eval `ssh-agent`
   ssh-add ~/.ssh/id_mfthomps_gfe
\end{verbatim}

\item You should now be able to ssh to the RESim server, e.g., 
\begin{verbatim}
   ssh -Y bladet9
\end{verbatim}
(Using -X seems to eventually time out because of temporary permissions?)

\end{itemize}

\subsection{IDA}
Configure IDA for remote use:
\begin{itemize}
\item Clone the RESim repo onto the computer from which you'll run IDA (need this for RESim
IDA Python plugins)

\item On the computer upon which you'll run IDA, create a ssh tunnel for use by the GDB port:
\begin{verbatim}
 ssh -fN -L 9123:localhost:9123 -oStrictHOstKeyChecking=no -oUserKnownHostsFile=/dev/null bladet9
\end{verbatim}

\item Start Ida and configure the debugger to use gdb, localhost and port 9123
\end{itemize}

\section{Configure RESim Server}
See section \ref{section-system} if this RESim server has not yet been configured.

\textbf{Steps below are implemented in RESim/simics/setup/config-resim-user.sh}
The following steps are taken for each user on the RESim server.
\begin{itemize}
\item Configure your git on RESim server to use a proxy:
\begin{verbatim}
    git config --global http.proxy http://webproxy:3128
    git config --global https.proxy https://webproxy:3128
\end{verbatim}

\item Add to your .bashrc:
\begin{verbatim}
    export http_proxy=http://webproxy:3128
    export https_proxy=https://webproxy:3128
    export RESIM=~/git/RESIM
\end{verbatim}
\item Start a new shell to inherit those variables:
\begin{verbatim}
   bash
\end{verbatim}

\item Use git to clone the RESim repo on the RESim server, e.g., 
\begin{verbatim}
   mkdir ~/git
   cd ~/git
   git clone https://github.com/mfthomps/RESim.git
\end{verbatim}

\item Create a "workspace" directory, and cd to it.  Then initialize
it as a Simics workspace:
\begin{verbatim}
    /mnt/simics/simics-4.8/simics-4.8.170/bin/workspace-setup
\end{verbatim}


\item Copy the files in git/RESim/simics/workspace to your workspace, and follow the README instructions.
\end{itemize}



\section{System Setup} 
\label{section-system}
These steps are only required once for each new server.
\textbf{Steps below are implemented in RESim/simics/setup/config-resim-server.sh}


\begin{itemize}
\item Add the following to the server /etc/hosts:
\begin{verbatim}
   10.20.200.41 webproxy
\end{verbatim}

\item Modify /etc/apt/apt.conf to include:
\begin{verbatim}
   Acquire::http::Proxy "http://webproxy:3128";
\end{verbatim}

\item Confirm the /etc/apt/sources.list refers to the proper mirror, e.g., 
\begin{verbatim}
    deb http://us.archive.ubuntu.com/ubuntu trusty universe
    deb http://us.archive.ubuntu.com/ubuntu trusty main restricted
    deb http://us.archive.ubuntu.com/ubuntu trusty-updates main restricted
\end{verbatim}


\item Create a mount point and add entry to the /etc/fstab:
\begin{verbatim}
    sudo mkdir /mnt/re_images
    sudo chmod a+rwx /mnt/re_images/
    Add to /etc/fstab:  webproxy:/ubuntu_img /mnt/re_images nfs4 auto 0 0
\end{verbatim}

\item Create link to shared images:
\begin{verbatim}
    sudo mkdir /eems_images
    cd /eems_images
    sudo ln -s /mnt/re_images ubuntu_img
\end{verbatim}


\item Install python-magic from gz file:  pip install <path>
\begin{verbatim}
   sudo pip install /mnt/re_images/python_pkgs/python-magic-0.4.15.tar.gz
\end{verbatim}

\item Install xterm
\begin{verbatim}
    apt-get install xterm
\end{verbatim}

\item Install git
\begin{verbatim}
    apt-get install git
\end{verbatim}
\end{itemize}

\section{Configure Simics licenses}
\textbf{These steps are automated in the RESim/setup/config-simics.sh script.}
\begin{itemize}
\item Get the Simics license server running  (name the license file that matches your ethernet MAC address:
\begin{verbatim}
   ./simics-gui -license-file /mnt/simics/simics-4.8.75/licenses/24B6FDF7BB64.lic
\end{verbatim}

\item Then quit.
Use 
\begin{verbatim}
   ps aux | grep lmgrd 
\end{verbatim}
to confirm

\item Install the vmx kernel module (Simics VMP)
\begin{verbatim}
   bin/vmp-kernel-install
   (follow instructions to enable on reboot)
\end{verbatim}
\end{itemize}
\end{document}
